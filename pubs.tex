\begin{etaremune}
  \item Rodriguez-Romaguera J\textsuperscript{\textasteriskcentered}, \textbf{Ung RL\textsuperscript{\textasteriskcentered}}, Nomura H\textsuperscript{\textasteriskcentered}, Otis J, Basiri M, Namboodiri VMK, Zhu X, Robinson JE, McHenry J, Kosyk O, Jhou T, Kash T, Bruchas M, Stuber GD (2020). Prepronociceptin expressing neurons in the extended amygdala encode and promote rapid arousal responses to motivationally salient stimuli. Accepted in \textbf{\textit{Cell Reports}}.
  \item Resendez SL, Namboodiri VMK, Otis JM, Eckman LEH, Rodriguez-Romaguera J, \textbf{Ung RL}, Basiri ML, Kosyk O, Rossi MA, Dichter GS, Stuber GD (2020). Social stimuli induce activation of oxytocin neurons within the paraventricular nucleus of the hypothalamus to promote social behavior in Male mice. \textbf{\textit{Journal of Neuroscience}}.
  \item Otis JM\textsuperscript{\textasteriskcentered}, Zhu MH\textsuperscript{\textasteriskcentered}, Namboodiri VMK, Cook CA, Kosyk O, Matan AM, Ying R, Hashikawa Y, Hashikawa K, Trujillo-Pisanty I, Guo J, \textbf{Ung RL}, Rodriguez-Romaguera J, Anton ES, Stuber GD (2019). Paraventricular Thalamus Projection Neurons Integrate Cortical and Hypothalamic Signals for Cue-Reward Processing. \textbf{\textit{Neuron}}.
  \item Resendez SL, Jennings JH, \textbf{Ung RL}, Namboodiri VMK, Zhou ZC, Otis JM, Nomura H, McHenry JA, Kosyk O, Stuber GD (2016). Visualization of cortical, subcortical and deep brain neural circuit dynamics during naturalistic mammalian behavior with head-mounted microscopes and chronically implanted lenses. \textbf{\textit{Nature Protocols}}.
  \item Jennings JH\textsuperscript{\textasteriskcentered}, \textbf{Ung RL\textsuperscript{\textasteriskcentered}}, Resendez SL, Stamatakis AM, Taylor JG, Huang J, Veleta K, Kantak PA, Aita M, Shilling-Scrivo K, Ramakrishnan C, Deisseroth K, Otte S, Stuber GD (2015). Visualizing hypothalamic network dynamics for appetitive and consummatory behaviors. \textbf{\textit{Cell}}.
  \item Sparta DR, Smithuis J, Stamatakis AM, Jennings JH, Kantak PA, \textbf{Ung RL}, Stuber GD (2014). Inhibition of projections from the basolateral amygdala to the entorhinal cortex disrupts the acquisition of contextual fear. \textbf{\textit{Frontiers in Behavioral Neuroscience}}.
  \item Sparta DR, Hovels{\o} N, Mason AO, Kantak PA, \textbf{Ung RL}, Decot HK, Stuber GD (2014). Activation of prefrontal cortical parvalbumin interneurons facilitates extinction of reward-seeking behavior. \textbf{\textit{Journal of Neuroscience}}.
  \item Sparta DR\textsuperscript{\textasteriskcentered}, Jennings JH\textsuperscript{\textasteriskcentered}, \textbf{Ung RL}, Stuber GD (2013). Optogenetic strategies to investigate neural circuitry engaged by stress. \textbf{\textit{Behavioural Brain Research}}.
  \item Stamatakis AM, Jennings JH, \textbf{Ung RL}, Blair GA, Weinberg RJ, Neve RL, Boyce F, Mattis J, Ramakrishnan C, Deisseroth K, Stuber GD (2013). A Unique Population of Ventral Tegmental Area Neurons Inhibits the Lateral Habenula to Promote Reward. \textbf{\textit{Neuron}}.
  \item Jennings JH, Rizzi G, Stamatakis AM, \textbf{Ung RL}, Stuber GD (2013). The inhibitory circuit architecture of the lateral hypothalamus orchestrates feeding. \textbf{\textit{Science}}.
  \item Jennings JH\textsuperscript{\textasteriskcentered}, Sparta DR\textsuperscript{\textasteriskcentered}, Stamatakis AM, \textbf{Ung RL}, Pleil KE, Kash TL, Stuber GD (2013). Distinct extended amygdala circuits for divergent motivational states. \textbf{\textit{Nature}}.
\end{etaremune}
